\documentclass[12pt]{article}
\usepackage[frenchb]{babel} 
\usepackage[T1]{fontenc}
\usepackage[latin1]{inputenc}
%\usepackage{lmodern} 
\usepackage{graphicx}
\usepackage{multirow}
\usepackage[top=2.5cm, bottom=2.5cm, left=2.5cm , right=2.5cm]{geometry}
%\usepackage{amsmath}
%\usepackage{amsthm}
%\usepackage{amsfonts}
\usepackage{empheq}
\usepackage{setspace}
\usepackage{hyperref}
\hypersetup{pdftitle = {INF4215 - Rapport de laboratoire}, pdfauthor={Julien Antoine}}
\usepackage{color}
\usepackage{subfigure}
\usepackage{fancyvrb}
\usepackage{SIunits}
\usepackage{numprint}
\usepackage{enumitem}
\usepackage{calc}
\usepackage{listings}
\usepackage{float}
\usepackage{cellspace}
\cellspacetoplimit=4pt
\cellspacebottomlimit=4pt

% ----------------------------------- FANCY HEADER -----------------------------------
\usepackage{fancyhdr}
\pagestyle{fancy}
\renewcommand{\headrulewidth}{0.5pt}
%\fancyhead[C]{\textbf{page \thepage}} 
\fancyhead[L]{}
\fancyhead[R]{Rapport de laboratoire}

\renewcommand{\footrulewidth}{0.5pt}
\fancyfoot[C]{\textbf{\thepage}} 
\fancyfoot[L]{Polytechnique Montr�al}
\fancyfoot[R]{INF4215 -- Intelligence artificielle}
% ------------------------------------------------------------------------------------


\providecommand{\e}[1]{\ensuremath{\cdot 10^{#1}}}
\newcommand{\question}{\noindent$\bullet$\;\;}
\newcommand{\eau}{\ensuremath{\text{H}_2 \text{O}}}
\newcommand{\dio}{\ensuremath{\text{CO}_2}}
%\addto\captionsfrancais{\renewcommand{\chaptername}{Labo}}

\begin{document}

\hyphenation{HyperLogLog experimental techno-logy according develop-ment}


\begin{titlepage}
\newcommand{\HRule}{\rule{\linewidth}{0.5mm}} % Defines a new command for the horizontal lines, change thickness here
\center % Center everything on the page
%------------------------------------------------------------------------------------
%	LOGO SECTION
%------------------------------------------------------------------------------------
\raggedright
\includegraphics[width = 0.33\textwidth]{../../../logo}\\[6cm] 
\centering
%------------------------------------------------------------------------------------
%	TITLE SECTION
%------------------------------------------------------------------------------------
\HRule \\[0.4cm]
{ \huge \bfseries INF4215 -- Rapport de laboratoire}\\[0.4cm]
{ \Large \bfseries Recherche dans un espace d'�tats}\\%[0.4cm] 
\HRule \\[1cm]
%------------------------------------------------------------------------------------
%	AUTHOR SECTION
%------------------------------------------------------------------------------------
\begin{minipage}{0.45\textwidth}
\begin{center} \large
Julien \textsc{Antoine}\\
1813026
\end{center}
\end{minipage}
~
\begin{minipage}{0.45\textwidth}
\begin{center} \large
Abdelhadi \textsc{Temmar}\\
1805815
\end{center}
\end{minipage}\\[8cm]
%------------------------------------------------------------------------------------
%	DATE SECTION
%------------------------------------------------------------------------------------
\begin{center}
{\large 14 f�vrier 2015}
\end{center}
%------------------------------------------------------------------------------------
\vfill % Fill the rest of the page with whitespace
\end{titlepage}

\section{Recherche arborescente}
\subsection{Repr�sentation du probl�me}
Chaque �tat contient la liste des points restant � couvrir, la liste des antennes plac�es et leur co�t total. A l'�tat initial, on a donc la liste compl�te des points � couvrir, une liste vide pour les antennes, et un co�t de 0.

\subsection{Recherche en profondeur}

\subsection{Recherche au moindre co�t}

\subsection{Recherche A*}


\section{Recherche locale}
Un deuxi�me type de recherche est la recherche locale. Contrairement � la recherche arborescente vue dans la section pr�c�dente, la locale part d'un �tat et essaie de l'am�liorer sur base de satisfaction de contraintes.

\subsection{Repr�sentation du probl�me}



\subsection{M�thode}



\section{R�ponses aux questions}
\paragraph{a)}



\paragraph{b)}











\end{document}
